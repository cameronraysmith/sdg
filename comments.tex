%%-----------------------------------------------------------------
%% PENDING / WORK IN PROGRESS
%\newcommand{\TODOitem}{%
%\typeout{WARNING!!! there is still a TODO left}
%\color{blue}\item
%}
%\newcommand{\TODO}[1]{%
%\typeout{WARNING!!! there is still a TODO left}
%\marginpar{\textbf{!TODO: }\emph{#1}}
%}
%
%\renewcommand{\TODO}[1]{}
%\nochangebars

%%-----------------------------------------------------------------
%% CATEGORIES
%\newcommand{\catSet}{\ensuremath{\textbf{Set}\xspace}}
%\newcommand{\catVect}{\ensuremath{\textbf{Vect}\xspace}}
%\newcommand{\catTop}{\ensuremath{\textbf{Top}\xspace}}
%\newcommand{\catPreGraphic}{\ensuremath{\textbf{PreGraphic}}\xspace}
%\newcommand{\catGraphic}{\ensuremath{\textbf{OGraphic}}\xspace}
%\newcommand{\catPreOGraph}{\ensuremath{\textbf{PreOGraph}}\xspace}
%\newcommand{\catOGraph}{\ensuremath{\textbf{OGraph}}\xspace}
%\newcommand{\catGraph}{\ensuremath{\textbf{Graph}}\xspace}
%\newcommand{\catModG}{\ensuremath{\mathsf{Mod}(\mathcal{G})}\xspace}
%\newcommand{\catGThy}{\ensuremath{\textbf{GThy}}\xspace}
%\newcommand{\slicecat}[2]{#1 / #2}
%\newcommand{\catGraphSlice}{\ensuremath{\slicecat{\catGraph}{2_\mathcal{G}}}}
%\newcommand{\catTGSlice}{\ensuremath{\slicecat{\catGraph}{T_\mathcal{G}}}}
%\newcommand{\catOGraphTG}{\ensuremath{\catOGraph_{\typegraph}}}
%\newcommand{\catMonPreCat}{\ensuremath{\textbf{MonPreCat}}\xspace}
%\newcommand{\rewriteCat}[1]{\ensuremath{\DCsp(\catOGraphTG)/\!\!/\,#1}}

%%-----------------------------------------------------------------
%% SPECIAL MATH
%\newcommand{\typegraph}{\ensuremath{T_{\mathcal G}}}
%\newcommand{\iso}{\cong}
%\newcommand{\denote}[1]{\llbracket #1 \rrbracket}
%%\newcommand{\Int}[1]{\mathop{\text{Int}}(#1)}
%%\newcommand{\Bry}[1]{\mathop{\text{Bry}}(#1)}
%\newcommand{\Bry}[1]{B_{#1}}
%\newcommand{\Int}[1]{\mathop{\text{Int}}_{#1}}
%% \sizeof{x} == |x|
%\newcommand{\sizeof}[1]{\left|#1\right|}
%\renewcommand{\split}[2]{#1_{{}\ltimes #2}}
%\newcommand{\plus}{+}
%\newcommand{\mergew}[1]{+_{#1}}
%\newcommand{\plugw}[1]{+_{\!#1}^{\!\!*}}
%\newcommand{\plug}{+^{\!\!*}}
%\newlength{\hookrightarrowwidth}
%\settowidth{\hookrightarrowwidth}{$\hookrightarrow$}%
%\DeclareMathOperator{\expandsto}{\stackrel{\prec}{\hookrightarrow}}
%
%\DeclareMathOperator{\matches}{%
%\makebox[\hookrightarrowwidth][c]{$\hookrightarrow$}
%\hspace*{-\hookrightarrowwidth}%
%\makebox[\hookrightarrowwidth][c]{\raise0.5mm\hbox{$\, ^{\sim}$}}%
%}
%
%\DeclareMathOperator{\leftmatches}{%
%\makebox[\hookrightarrowwidth][c]{$\hookleftarrow$}
%\hspace*{-\hookrightarrowwidth}%
%\makebox[\hookrightarrowwidth][c]{\raise0.5mm\hbox{$\, ^{\sim}$}}%
%}
%
%\DeclareMathOperator{\notleftmatches}{%
%\makebox[\hookrightarrowwidth][c]{$\hookleftarrow$}
%\hspace*{-\hookrightarrowwidth}%
%\makebox[\hookrightarrowwidth][c]{\raise0.5mm\hbox{$\, ^{\sim}$}}%
%\hspace*{-\hookrightarrowwidth}%
%\makebox[\hookrightarrowwidth][c]{\raise-0.2mm\hbox{\footnotesize{$/$}}}%
%}
%
%\newcommand{\seqcompww}[3]{#1\, ;_{#2} #3}
%\newcommand{\seqcompss}[2]{#1\, ;_{\mathit{p}} #2}
%\newcommand{\exten}[2]{{#1}^{\uparrow #2}}
%\newcommand{\tensor}{\otimes}
%\newcommand{\injmap}{\hookrightarrow}
%\newcommand{\linjmap}{\hookleftarrow}
%
%\newcommand{\In}{\textrm{In}}
%\newcommand{\Out}{\textrm{Out}}
%%\newcommand{\Bound}{\textrm{Bound}}
%%\newcommand{\Ext}{\textrm{FIXME!!!}}
%%\newcommand{\Intern}{\textrm{Intern}}
%%\newcommand{\Isol}{\textrm{Isol}}
%\newcommand{\DCsp}{\textbf{DCsp}}
%%\newcommand{\Interf}{\textrm{Interf}}
%\newcommand{\Homeo}[1]{\denote{#1}_\sim}
%\newcommand{\HomeoFun}{\denote{-}_\sim}


%%-----------------------------------------------------------------
%% Rewriting
%\newcommand{\cmdrewritesto}{\tikz[baseline=-0.25em] { \draw [-open triangle 45, line width=0.2pt] (0,0) -- (0.5,0); }\,}
%\newcommand{\cmdrewriteequiv}{\tikz[baseline=-0.25em] { \draw [open triangle 45-open triangle 45, line width=0.2pt] (0,0) -- node [auto,yshift=-1.2mm] {$*$} (0.7,0); }\,}
%\newcommand{\cmdrewritetrans}{\tikz[baseline=-0.25em] { \draw [-open triangle 45, line width=0.2pt] (0,0) -- node [auto,pos=0.3,yshift=-1.2mm] {$*$} (0.5,0); }\,}
%
%\newcommand{\lengthymultimapdot}[1]{%
%\begin{tikzpicture}[baseline=(current bounding box.south), node distance=0.5em and 1em,text height=0em, text depth=0ex,inner sep=0em]
%  \node[fill=white, line width=0.2pt, draw=black, inner sep=1.5pt, circle] (b) {};
%  \path[-] ($(b) - (#1,0)$) edge (b);
%\end{tikzpicture}%
%}
%%\newcommand{\longmultimapdot}{\,\lengthymultimapdot{1.7em}\xspace}
%%\newcommand{\multimapdot}{\,\lengthymultimapdot{1em}\xspace}
%
%\DeclareMathOperator{\multimapdot}{\lengthymultimapdot{1em}}
%\DeclareMathOperator{\longmultimapdot}{\lengthymultimapdot{1.7em}}
%
%\DeclareMathOperator{\rewritesto}{\cmdrewritesto}
%\DeclareMathOperator{\rewriteequiv}{\cmdrewriteequiv}
%\DeclareMathOperator{\rewritetrans}{\cmdrewritetrans}
%
%
%%\newcommand{\multimapdot}{-\!\!\!\bullet}
%%\newcommand{\longmultimapdot}{\multimapdot\multimapdot}
%%% rewrite arrows
%\newcommand{\rwarrow}{\multimapdot}
%\newcommand{\longrwarrow}{\longmultimapdot}
%%% rewrite rule with boundary data
%\newcommand{\rwrulew}[3]{#1 \multimapdot_{#2} #3}
%%% Rewrite rule without boundary data
%\newcommand{\rwrule}[2]{#1 \multimapdot #2}
%%% substitution with boundary data
%\newcommand{\rwsubstw}[3]{#1 \multimapdot_{#2} #3}
%%% substitution without boundary data
%\newcommand{\rwsubst}[2]{#1 \multimapdot #2}
%%% ???
%%\newcommand{\xrwsubst}[2]{#1 \multimapdot #2}

%%-----------------------------------------------------------------
%% THEOREM ENV
%\theoremstyle{plain} % default 
%\newtheorem{theorem}{Theorem}[section]
%\newtheorem{proposition}[theorem]{Proposition}
%\newtheorem{lemma}[theorem]{Lemma}
%\newtheorem{corollary}[theorem]{Corollary}
%\newtheorem{conjecture}[theorem]{Conjecture}
%
%%\theoremstyle{definition}
%\newtheorem{definition}[theorem]{Definition}
%\newtheorem{definitions}[theorem]{Definitions}
%
%%\theoremstyle{remark}
%\newtheorem{example}[theorem]{Example}
%\newtheorem{examples}[theorem]{Examples}
%\newtheorem{remark}[theorem]{Remark}